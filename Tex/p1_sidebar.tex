\begin{CVsidebar}

\section{About me}
Age: \softtext{24 years old, born 31.01.2000} \par
Status: \softtext{Doctoral student, Assistant professor at the University of Lausanne} \par
Interrests: \softtext{Chess 1950 Elo;}\\ \softtext{Climbing 7A;} \\ \softtext{Pianist, Organist, Cellist, Composer and
Concertist
(CEM, DEM, more than 17 years of
practicing)
} \par
Transports: \softtext{B License}

\vspace{\baselineskip}
\begin{quote}
    Passions do less harm than boredom, for passions always tend to decrease, while boredom always tends to increase.
\end{quote}
% NOTE: you can change the quote colour 
% by adding a colour in square brackets [ ]:
% \begin{quote}[red] will make the quote
% appear in red. Similarly for soft_text and
% other colours defined in the preamble or .cls

\section{Languages}

% Options for displaying languages:
%-------------------------------
% * \LngLvlCEFR{<Language>}{<Level>}:
%      Specify CEFR language <Level>: 
%      A1, A2, B1, B2, C1, C2 or 0
%
% * \LngLvlILR{<Language>}{<Level>}:
%       Specify ILR language <Level>:
%       0, 1, 2, 3, 4 or 5
%
% * \CVrating[<text>]{<Language>}{<level>}{<max level>}:
%       <text> will appear before rating circles
%       it will display <level>/<max level> circles,
%       i.e. if you specify 2 and 3 it will show 
%       2 green circles out of a total of 3 circles.
%
% * \LngLvlWords{<Language>}{<text>}:
%       appears as <Language> \hfill \textit{<text>}
%       good for expressing language proficiency with words
%       i.e. " English      Professional Working knowledge "
%       or   " English                   Intermidiate User "
%
% * \LngTextCFR :
%       prints text:
%       " *Using CEFR rating for language proficency "
%       in current document language
%
% * \LngTextILR :
%       prints text:
%       " *Using ILR rating for language proficency "
%       in current document language
%
% Comment: Preferably only use one style or you may combine
%       \LngLvlCEFR / \LngLvlILR with \LngLvlWords in  this manner:
%       \LngLvlCEFR{<language>}{<Level>} and \LngLvlWords{}{<description>}

\LngLvlWords{Below, the dots represent my daily use of each concept}{\newline}

\LngLvlCEFR{French}{C2}
\LngLvlWords{}{Native}

\CVrating{English}{4}{5}
\LngLvlWords{}{TOEIC 960}
\CVrating{German}{2}{5}
\LngLvlWords{}{B2 Baccalaureat level}

%% For unsupported languages for \LngTextCEFR and \LngTextILR, use this:
% \vspace{0.5\baselineskip}
% {\footnotesize\softtext{*Using CEFR rating for language proficency.}}

\dotline

\CVrating{Python}{5}{5}
\CVrating{\LaTeX/Shell}{4}{5}
\CVrating{C/C\#/C++/Java/HTML5/CSS/SQL }{2}{5}

\dotline

\CVrating{Pandas/Numpy}{5}{5}
\CVrating{Sklearn/Tensorflow/Pytorch/Dash}{5}{5}
\CVrating{Django/Tkinter}{3}{5}

\section{Tools I use}
% You can use itemlist instead of itemize.
% This has less space between items.
\begin{itemlist}
    \item \textbf{Organization} \href{https://obsidian.md/}{Obsidian} \& \href{https://www.notion.so/}{Notion} \& \href{https://voicenotes.com/}{VoiceNotes}
    \item \textbf{Programming} : \href{https://www.cursor.com/}{Cursor}
\end{itemlist}



\end{CVsidebar}